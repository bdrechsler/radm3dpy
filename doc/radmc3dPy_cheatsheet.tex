\documentclass[12pt]{article}
\usepackage{graphicx}
\usepackage{amsmath}
\usepackage[left=1.5cm, right=1.5cm, top=1.5cm, bottom=1.5cm]{geometry}
\usepackage{color}
\usepackage{colortbl}
\usepackage{tikz}
\usetikzlibrary{shapes, shadows, arrows}


\definecolor{darkgreen}{RGB}{0,150,0}
\definecolor{lblue}{RGB}{200,200,250}
\definecolor{lgreen}{RGB}{200,250,200}
\definecolor{lred}{RGB}{250,200,200}
\definecolor{yellow}{RGB}{230,230,150}

\renewcommand{\rmdefault}{phv}
\newcommand{\pymod}{{\tt  radmc3dPy }}


\begin{document}

% Some Tikz definitions for the flowchart
\tikzstyle{decision} = [diamond, draw, fill=blue!50]
\tikzstyle{line} = [draw, -stealth, thick]
\tikzstyle{output}=[draw, ellipse, fill=green!50,minimum height=8mm, text width=7em, text centered]
\tikzstyle{calc}=[draw, ellipse, fill=red!40,minimum height=8mm, text width=10em, text centered]
\tikzstyle{input} = [draw, rectangle, fill=blue!30, text width=12em, text centered, minimum height=15mm, node distance=5em]
\tikzstyle{opinput} = [draw, rectangle, fill=yellow!50, text width=12em, text centered, minimum height=15mm, node distance=5em]



\begin{center}
{\huge\bf radmc3dPy v0.22}\\
\vspace{0.5cm}
{\large Attila Juh\'asz}
\end{center}

\section*{module: setup}
\begin{itemize}
\item[] {\bf get\_model\_desc()} - Returns the brief description of the model
\item[] {\bf get\_model\_names()} - Returns the list of available models
\item[] {\bf problem\_setup\_dust()} - Creats a dust continuum model setup
\item[] {\bf problem\_setup\_gas()} - Creates a gas model setup
\end{itemize}

\section*{module: analyze}
\begin{itemize}
\item[]{\bf read\_data()} - Reads variables (e.g. dust density, gas velocity, etc)
\item[]{\bf read\_grid()} - Reads the spatial and frequency grid
\item[]{\bf readopac()} - Reads the dust opacities
\item[]{\bf readparams()} - Reads the parameter file (problem\_params.inp)
\item[]{\bf write\_default\_parfile()} - Writes the default parameters for a model
\item[]{\bf radmc3dData.get\_sigmadust()} - Calculates the dust surface density in g/cm$^2$
\item[]{\bf radmc3dData.get\_sigmagas()} - Calculates the gas surface density in molecule/cm$^2$
\item[]{\bf radmc3dData.get\_tau()} - Calculates the continuum optical depth 
\item[]{\bf radmc3dData.write\_vtk()} - Writes variables to a VTK format for visualisation with e.g. Paraview 
\end{itemize}

\section*{module: image}
\begin{itemize}
\item[]{\bf makeimage()} - Calculates an image with RADMC-3D (both dust continuum and channel maps)
\item[]{\bf plotimage()} - Plot the image / channel map
\item[]{\bf readimage()} - Reads the image
\item[]{\bf radmc3dImage.imconv()} - Convolve the image with a Gaussian beam 
\item[]{\bf radmc3dImage.plot\_momentmap()} - Plots moment map for a 3d image cube
\item[]{\bf radmc3dImage.writefits()} - Writes the image to a FITS file with CASA compatible header
\end{itemize}

\newpage
\section*{Dust continuum model} 

\begin{figure}[!hp]
\begin{center}
\begin{tikzpicture}

% Input nodes
\node [input] (grid) {{\bf Spatial \& Frequency grid}\\amr\_grid.inp, \\wavelength\_micron.inp};
\node [input, below of=grid, yshift=-2.em] (radsource) {{\bf Radiation source(s)} \\ stars.inp};
\node [input, below of=radsource, yshift=-2.em] (dustopac) {{\bf Dust opacity} \\ dustkappa\_xxx.inp, dustopac.inp};
\node [input, below of=dustopac, yshift=-2.em] (dustdens) {{\bf Dust density} \\ dust\_density.inp};
\node [input, below of=dustdens, yshift=-2.em] (codepar) {{\bf Code parameters} \\ radmc3d.inp};

% MC calculation
\node [calc, right of=dustopac, xshift=13.em] (thermalmc) {{\bf Themal Monte-Carlo ($\rightarrow$ T$_{\rm dust}$)} \\ dust\_temperature.dat};

% Observables
\node [output, right of=thermalmc, xshift=13.em, yshift=5em] (contsed) {{\bf Raytracing \\ SED} \\ spectrum.out};
\node [output, right of=thermalmc, xshift=13.em, yshift=-5em] (contimag) {{\bf Raytracing \\ Image} \\ image.out};

% Lines/arrows
\path [line] (grid) -| node[xshift=3em, yshift=-10em] {} (thermalmc);
\path [line] (radsource) -| node[xshift=1em, yshift=-10em] {} (thermalmc);
\path [line] (dustopac) -- (thermalmc);
\path [line] (dustdens) -| node[xshift=3em, yshift=10em] {} (thermalmc);
\path [line] (codepar) -| node[xshift=1em, yshift=10em] {} (thermalmc);

\path [line] (thermalmc) -- (contsed);
\path [line] (thermalmc) -- (contimag);

\end{tikzpicture}
\end{center}
\caption{Structure of a dust continuum model. Inputs are marked with blue, intermediate calculation, data products are
in red while green marks the output of the simulation.}
\end{figure}

\subsection*{radmc3dPy commands}
First let us create a directory for our model. Then go to this directory and start python. 

\begin{itemize}
\item[1] Import \pymod.

{\tt >>> import radmc3dPy}\\
\item[2] Check which models are available:

{\tt >>> radmc3dPy.setup.get\_model\_names()}\\
\indent {\tt ['ppdisk']}\\
\item[3] Create a parameter file with the default values

{\tt >>> radmc3dPy.analyze.write\_default\_parfile('ppdisk')}\\

Exit python and open the created 'problem\_params.inp' file with a text editor and change the parameters if needed. 
\item[4] Set up the model and create all necessary input files.

{\tt>>>radmc3dPy.setup.problem\_setup\_dust('ppdisk')}\\
\item[5] Then run RADMC-3D from the shell with the Monte-Carlo simulation to calculate the dust temperature.

{\tt \$>radmc3d mctherm}\\

Alternatively we can also make a system call from within Python, e.g.:

{\tt>>>os.system('radmc3d mctherm')}
\item[6] After the thermal Monte-Carlo run has finished we can make an image from within pyton. 

{\tt \$>radmc3dPy.image.makeimage(npix=400, sizeau=200, wav=880., incl=45, posang=43.)}\\
\item[7] After RADMC-3D finished we can read the image and plot it. 

{\tt>>>imag=radmc3dPy.image.readimage()}\\
{\tt>>>radmc3dPy.image.plotimage(imag)}

\item[8] We can also convolve the image with an arbitrary elliptical gaussian Gaussian beam

{\tt>>>conv\_imag = imag.imconv(fwhm=[0.05, 0.1], pa=40., dpc=140.)}

The fwhm of the Gaussian beam should be in arcsec,  the positing angle of the major axis of the beam ellipse
should be in degrees, and the distance to the source in pc (dpc keyword) should be given. 


\end{itemize}


\newpage
\section*{Gas model} 

\begin{tikzpicture}

% Input nodes
\node [input] (grid) {{\bf Spatial \& Frequency grid}\\amr\_grid.inp, \\wavelength\_micron.inp};
\node [input, below of=grid, yshift=-1.em] (radsource) {{\bf Radiation source(s)} \\ stars.inp};
\node [input, below of=radsource, yshift=-1.em] (dustopac) {{\bf Dust opacity} \\ dustkappa\_xxx.inp, dustopac.inp};
\node [input, below of=dustopac, yshift=-1.em] (dustdens) {{\bf Dust density} \\ dust\_density.inp};
%\node [input, below of=dustdens, yshift=-1.em] (dusttemp) {{\bf Dust temperature} \\ dust\_temperature.dat};
\node [input, below of=dustdens, yshift=-2.em] (codepar) {{\bf Code parameters} \\ radmc3d.inp};

\node [opinput, below of=codepar, yshift=-1.em] (gastemp) {{\bf Gas temperature} \\ gas\_temperature.dat};
\node [opinput, below of=gastemp, yshift=-1.em] (velocity) {{\bf Velocity} \\ gas\_velocity.inp};
\node [opinput, below of=velocity, yshift=-1.em] (gasdens) {{\bf Gas density} \\ numberdens\_xxx.inp};
\node [opinput, below of=gasdens, yshift=-1.em] (moldata) {{\bf Molecular data} \\ molecule\_xxx.inp};
\node [opinput, below of=moldata, yshift=-1.em] (linesinp) {{\bf Line RT setup} \\ lines.inp};



% MC calculation
\node [calc, right of=dustopac, xshift=13.em] (thermalmc) {{\bf Themal Monte-Carlo ($\rightarrow$ T$_{\rm dust}$)} \\ dust\_temperature.dat};
% Excitation calculation
\node [calc, right of=thermalmc, xshift=-2.4em, yshift=-28.em] (excitation) {{\bf Excitation \\ ($\rightarrow$ Level pop.)} \\ };

% Observables
\node [output, right of=thermalmc, xshift=13.em, yshift=5em] (contsed) {{\bf Raytracing \\ SED} \\ spectrum.out};
\node [output, right of=thermalmc, xshift=13.em, yshift=-5em] (contimag) {{\bf Raytracing \\ Image} \\ image.out};

\node [output, right of=excitation, xshift=13.em, yshift=5em] (spectrum) {{\bf Raytracing \\ Spectrum} \\ spectrum.out};
\node [output, right of=excitation, xshift=13.em, yshift=-5em] (cmap) {{\bf Raytracing \\ Channel map} \\ image.out};

% Lines/arrows
\path [line] (grid) -| node[xshift=3em, yshift=-10em] {} (thermalmc);
\path [line] (radsource) -| node[xshift=1em, yshift=-10em] {} (thermalmc);
\path [line] (dustopac) -- (thermalmc);
\path [line] (dustdens) -| node[xshift=3em, yshift=10em] {} (thermalmc);
\path [line] (codepar) -| node[xshift=1em, yshift=10em] {} (thermalmc);

\path [line] (thermalmc) -- (contsed);
\path [line] (thermalmc) -- (contimag);


%\path [line] (gastemp) -- (excitation);
\path [line] (gastemp) -| node[xshift=3em, yshift=-10em] {} (excitation);
\path [line] (velocity) -| node[xshift=3em, yshift=-10em] {} (excitation);
\path [line] (gasdens) -| node[xshift=3em, yshift=10em] {} (excitation);
\path [line] (moldata) -| node[xshift=3em, yshift=10em] {} (excitation);
\path [line] (linesinp) -| node[xshift=3em, yshift=10em] {} (excitation);
\path [line] (thermalmc) -- (excitation);


\path [line] (excitation) -- (spectrum);
\path [line] (excitation) -- (cmap);
\path [line] (thermalmc) -- (spectrum);
\path [line] (thermalmc) -- (cmap);


\end{tikzpicture}


\subsection*{radmc3dPy commands}
\begin{itemize}

\item[1] Import \pymod.

{\tt >>> import radmc3dPy}\\
\item[2] Check which models are available:

{\tt >>> radmc3dPy.setup.get\_model\_names()}\\
\indent {\tt ['ppdisk']}\\
\item[3] Create a parameter file with the default values

{\tt >>> radmc3dPy.analyze.write\_default\_parfile('ppdisk')}\\
\item[4] Exit python and open the created 'problem\_params.inp' file with a text editor and change the parameters if needed. 
Create all necessary imput files.

{\tt>>>radmc3dPy.setup.problem\_setup\_gas('ppdisk')}\\
\item[5] This time we can skip the thermal Monte-Carlo simulation and go directly to raytracing, calculating images
and spectra.

{\tt \$>radmc3d image npix 400 sizeau 200 incl 45. phi 0. posang 43. iline 3 vkms 1.0}\\
\item[6] This command calculates a single channel map at 

{\tt>>>imag=radmc3dPy.image.readimage()}\\
{\tt>>>radmc3dPy.image.plotimage()}

\end{itemize}









\end{document}